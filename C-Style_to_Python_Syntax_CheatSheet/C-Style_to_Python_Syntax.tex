\documentclass{article}

\usepackage{color}
\usepackage[margin=0.5in]{geometry}
\usepackage{listings}

\definecolor{dkgreen}{rgb}{0,0.6,0}
\definecolor{gray}{rgb}{0.5,0.5,0.5}
\definecolor{mauve}{rgb}{0.58,0,0.82}

\lstset{
  language=Python,
  basicstyle={\small\ttfamily},
  breakatwhitespace=true,
  breaklines=true,
  columns=flexible,
  commentstyle=\color{dkgreen},
  escapeinside={(*@}{@*)},
  morekeywords={as, None, with, yield, True, False},
  keywordstyle=\color{blue},
  numbers=left,
  showstringspaces=false,
  stringstyle=\color{mauve},
  tabsize=2,
  emph={self},
  emphstyle=\color{red}
}

\begin{document}

\section{I/O}
\begin{minipage}{0.45\linewidth}
    Java:
\end{minipage}
\hfill
\begin{minipage}{0.45\linewidth}
    Python:
\end{minipage}

\begin{minipage}{0.45\linewidth}
    \begin{lstlisting}[language=Java]
System.out.println("Hello world");
System.out.println("Earth is #" + 1);
int num = sc.nextInt();  // Assuming that you've done the appropriate overhead
    \end{lstlisting}
\end{minipage}
\hfill
\begin{minipage}{0.45\linewidth}
    \begin{lstlisting}
print("Hello World")
print("Earth is #{}".format(1))  # Similar to printf
num = int(input('Enter a number: '))  # There is no overhead
    \end{lstlisting}
\end{minipage}

\section{if, else, and elif}
\begin{minipage}{0.45\linewidth}
    Java:
\end{minipage}
\hfill
\begin{minipage}{0.45\linewidth}
    Python:
\end{minipage}

\begin{minipage}{0.45\linewidth}
    \begin{lstlisting}[language=Java]
if(CONDITIONAL) {
    Stuff...
}
else if(CONDITIONAL) {
    Stuff...
}
else {
    Stuff...
}
    \end{lstlisting}
\end{minipage}
\hfill
\begin{minipage}{0.45\linewidth}
    \begin{lstlisting}
if CONDITIONAL:
    Stuff...
elif CONDITIONAL:
    Stuff...
else:
    Stuff...
    \end{lstlisting}
\end{minipage}

\section{Loops}
\subsection{while}
\begin{minipage}{0.45\linewidth}
    Java:
\end{minipage}
\hfill
\begin{minipage}{0.45\linewidth}
    Python:
\end{minipage}

\begin{minipage}{0.45\linewidth}
    \begin{lstlisting}[language=Java]
while(CONDITIONAL) {
    Stuff on repeat...
}
    \end{lstlisting}
\end{minipage}
\hfill
\begin{minipage}{0.45\linewidth}
    \begin{lstlisting}
while CONDITIONAL:
    Stuff on repeat...
    \end{lstlisting}
\end{minipage}

\subsection{C-Style for}
\begin{minipage}{0.45\linewidth}
    Java:
\end{minipage}
\hfill
\begin{minipage}{0.45\linewidth}
    Python:
\end{minipage}

\begin{minipage}{0.45\linewidth}
    \begin{lstlisting}[language=Java]
for(int i = 0; i < n; ++i) {
    Stuff on repeat n times...
}

for(int i = 0; i < a.length; ++i) {
    Use a[i]...
}
    \end{lstlisting}
\end{minipage}
\hfill
\begin{minipage}{0.45\linewidth}
    \begin{lstlisting}
for i in range(n):
    Stuff on repeat n times...

for i in range(len(a)):
    Use a[i]...
    \end{lstlisting}
\end{minipage}

\subsection{Python Style}
\begin{lstlisting}
for ai in a:
    Stuff using ai (*@$\equiv$@*) a[i]...

OR

for i, ai in enumerate(a):
    Stuff using ai (*@$\equiv$@*) a[i]...
\end{lstlisting}

\section{Booleans}
\begin{minipage}{0.45\linewidth}
    Java:
\end{minipage}
\hfill
\begin{minipage}{0.45\linewidth}
    Python:
\end{minipage}

\begin{minipage}{0.45\linewidth}
    \begin{lstlisting}[language=Java]
true
false
    \end{lstlisting}
\end{minipage}
\hfill
\begin{minipage}{0.45\linewidth}
    \begin{lstlisting}
True
False
    \end{lstlisting}
\end{minipage}

\begin{minipage}{0.45\linewidth}
    Java:
\end{minipage}
\hfill
\begin{minipage}{0.45\linewidth}
    Python:
\end{minipage}

\begin{minipage}{0.45\linewidth}
    \begin{lstlisting}[language=Java]
if(x > 1) {...
if(x >= 1) {...
if(x == 1) {...
if(x != 1) {...
if(!b) {...
if(b1 && (b2 || !b3)) {...
if(1 < x && x < 10) {...
if(x == y && y == z) {...
    \end{lstlisting}
\end{minipage}
\hfill
\begin{minipage}{0.45\linewidth}
    \begin{lstlisting}
if x > 1:...
if x >= 1:...
if x == 1:...
if x != 1:...
if not b:...
if b1 and (b2 or not b3):...
if 1 < x < 10:...
if x == y == z:...
    \end{lstlisting}
\end{minipage}

\section{Arithmetic}
\begin{minipage}{0.45\linewidth}
    Java:
\end{minipage}
\hfill
\begin{minipage}{0.45\linewidth}
    Python:
\end{minipage}

\begin{minipage}{0.45\linewidth}
    \begin{lstlisting}[language=Java]
x = 5;
x = x + 1;  // x == 6
x += 1;  // x == 6
x++;  // or ++x;  x == 6
x /= 2;  // x == 2
N/A  // Java does not support
N/A  // Java does not support
N/A  // Java does not support
x %= 2;  // x == 1
    \end{lstlisting}
\end{minipage}
\hfill
\begin{minipage}{0.45\linewidth}
    \begin{lstlisting}
x = 5
x = x + 1  # x == 6
x += 1  # x == 6
N/A  # Python does not support
x /= 2  # x == 2.5
x //= 2  # x == 2 (Integer division)
x = 5 ** 2  # 25 (5 * 5)
x **= 2  # x == 25 (x * x)
x %= 2  # x == 1
    \end{lstlisting}
\end{minipage}

\section{Methods / Functions}
\begin{minipage}{0.45\linewidth}
    Java:
\end{minipage}
\hfill
\begin{minipage}{0.45\linewidth}
    Python:
\end{minipage}

\begin{minipage}{0.45\linewidth}
    \begin{lstlisting}[language=Java]
ACCESS_MODIFIER RETURN_TYPE FUNCTION_NAME(PARAMETERS) {...

public int add(int x, int y) {
    return x + y;
}
    \end{lstlisting}
\end{minipage}
\hfill
\begin{minipage}{0.45\linewidth}
    \begin{lstlisting}
def FUNCTION_NAME(PARAMETERS):...

def add(x, y):
    return x + y
    \end{lstlisting}
\end{minipage}

\subsection{Overloading / Keyword Arguments}
\begin{minipage}{0.45\linewidth}
    Java:
\end{minipage}
\hfill
\begin{minipage}{0.45\linewidth}
    Python:
\end{minipage}

\begin{minipage}{0.45\linewidth}
    \begin{lstlisting}[language=Java]
public double calc(double x) {
    return calc(x, 5, 5);
}

public double calc(double x, double y, double z) {
    return x + (y / z);
}

...

calc(3);  // returns 4
calc(3, 4, 2);  // returns 5
    \end{lstlisting}
\end{minipage}
\hfill
\begin{minipage}{0.45\linewidth}
    \begin{lstlisting}
def calc(x, y=5, z=5):
    return x + (y / z)

...

calc(3)  # returns 4
calc(3, 4)  # returns 3.8
calc(3, 4, 2)  # returns 5
calc(3, z=4)  # returns 4.25
calc(3, y=8, z=4)  # returns 5
calc(3, z=4, y=8)  # returns 5
    \end{lstlisting}
\end{minipage}

\section{Classes}
\subsection{Class Declaration}
\begin{minipage}{0.45\linewidth}
    Java:
\end{minipage}
\hfill
\begin{minipage}{0.45\linewidth}
    Python:
\end{minipage}

\begin{minipage}{0.45\linewidth}
    \begin{lstlisting}[language=Java]
ACCESS_MODIFIER class CLASS_NAME extends PARENT_CLASS implements INTERFACES {...

public class MyClass extends ParentClass implements TheirInterface {...

public class MyClass {...  // Implicitely extends Object class
    \end{lstlisting}
\end{minipage}
\hfill
\begin{minipage}{0.45\linewidth}
    \begin{lstlisting}
class CLASS_NAME(PARENT_CLASS1, PARENT_CLASS2, etc):...

class MyClass(ParentClass):

class MyClass(object):  # Explicitely extends object class
    \end{lstlisting}
\end{minipage}

\subsection{Constructor and Methods}
\begin{minipage}{0.45\linewidth}
    Java:
\end{minipage}
\hfill
\begin{minipage}{0.45\linewidth}
    Python:
\end{minipage}

\begin{minipage}{0.45\linewidth}
    \begin{lstlisting}[language=Java]
public MyClass(int xIn, double yIn, boolean zIn) {
    // Assuming that MyClass extends a class that has a constructor with int x
    // Assuming that MyClass declared instance variables:
    //      double y;
    //      boolean z;
    super(xIn);

    y = yIn;  // (*@$\equiv$@*) this.y = yIn;
    z = zIn;  // (*@$\equiv$@*) this.z = zIn;
}

public get2y() {
    return y * 2;  // (*@$\equiv$@*) return this.y * 2;
}
     \end{lstlisting}
\end{minipage}
\hfill
\begin{minipage}{0.45\linewidth}
    \begin{lstlisting}
def __init__(self, x_in, y_in, z_in):
    # Assuming that MyClass extends a class that has a constructor with int x
    super().__init__(x_in)

    self.y = y_in  # (*@$\not\equiv$@*) y = y_in;
    self.z = z_in  # (*@$\not\equiv$@*) z = z_in;

def get2y(self):
    return self.y * 2  # (*@$\not\equiv$@*) return y * 2
    \end{lstlisting}
\end{minipage}

\subsection{Initialization and Method Calling}
\begin{minipage}{0.45\linewidth}
    Java:
\end{minipage}
\hfill
\begin{minipage}{0.45\linewidth}
    Python:
\end{minipage}

\begin{minipage}{0.45\linewidth}
    \begin{lstlisting}[language=Java]
MyClass mc = new MyClass(2, 3.5, true);
System.out.println(mc.get2y());
     \end{lstlisting}
\end{minipage}
\hfill
\begin{minipage}{0.45\linewidth}
    \begin{lstlisting}
mc = MyClass(2, 3.5, True)
print(mc.get2y())  # Note that while self is an parameter I don't actually have to pass it
    \end{lstlisting}
\end{minipage}

\section{Include / Import}
\begin{minipage}{0.45\linewidth}
    Java:
\end{minipage}
\hfill
\begin{minipage}{0.45\linewidth}
    Python:
\end{minipage}

\begin{minipage}{0.45\linewidth}
    \begin{lstlisting}[language=Java]
// Not needed if file is in the same directory

// Reference things in the package as thing
include path.to.folder.package;
    \end{lstlisting}
\end{minipage}
\hfill
\begin{minipage}{0.45\linewidth}
    \begin{lstlisting}
# Needed always

# Reference thing as path.to.folder.module.thing
import path.to.folder.module

# Reference things in module as md.thing
import path.to.folder.module as md

# Reference thing as thing
from path.to.folder.module import thing

# If it is in the same directory
import module
    \end{lstlisting}
\end{minipage}

\subsection{Examples}
\begin{minipage}{0.45\linewidth}
    Java:
\end{minipage}
\hfill
\begin{minipage}{0.45\linewidth}
    Python:
\end{minipage}

\begin{minipage}{0.45\linewidth}
    \begin{lstlisting}[language=Java]
include java.util.Scanner;

Scanner sc = new Scanner(System.in);
    \end{lstlisting}
\end{minipage}
\hfill
\begin{minipage}{0.45\linewidth}
    \begin{lstlisting}
# Don't worry about what this does, it's an
# example of how to import

# This is the only style of import that I will use
import scipy.sparse
matrix = scipy.sparse.csr_matrix(range(10))

import scipy.sparse as sp_sparse
matrix = sp_sparse.csr_matrix(range(10))

from scipy.sparse from csr_matrix
matrix = csr_matrix(range(10))
    \end{lstlisting}
\end{minipage}

\newpage

\section{Example Factorial Program}
\subsection{Java}
FactorialCalculartor.java
\lstinputlisting[language=Java]{FactorialCalculator.java}
Factorial.java
\lstinputlisting[language=Java]{Factorial.java}

\subsection{Python}
factorial\_calculator.py
\lstinputlisting[language=Python]{factorial_calculator.py}
factorial.py
\lstinputlisting[language=Python]{factorial.py}

\end{document}
