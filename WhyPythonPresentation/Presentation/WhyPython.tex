\documentclass[11pt]{beamer}
\usetheme{Copenhagen}
\usecolortheme{dolphin}

\usepackage[utf8]{inputenc}
\usepackage{amsmath}
\usepackage{amsfonts}
\usepackage{amssymb}
\usepackage{array}
\usepackage{caption}
\usepackage{datetime}
\usepackage{fancyhdr}
\usepackage{gensymb}
\usepackage{graphicx}
\usepackage{lastpage}
\usepackage{multicol}
\usepackage{multirow}
\usepackage{pgfpages}
\usepackage{setspace}
\usepackage{tikz}

% Code snippets
\usepackage{listings}
\usepackage{color}

\definecolor{dkgreen}{rgb}{0,0.6,0}
\definecolor{gray}{rgb}{0.5,0.5,0.5}
\definecolor{mauve}{rgb}{0.58,0,0.82}

\lstset{
  basicstyle={\small\ttfamily},
  breakatwhitespace=true,
  breaklines=true,
  columns=flexible,
  commentstyle=\color{dkgreen},
  escapeinside={(*@}{@*)},
  keywordstyle=\color{blue},
  numbers=none,
  numberstyle=\color{gray},
  showstringspaces=false,
  stringstyle=\color{mauve},
  tabsize=2,
  language=Python,
  morekeywords={as, None, with, yield}
}

\newcommand{\code}[1]{
\begin{lstlisting}
#1
\end{lstlisting}}

\newcommand{\emptyline}{$ $\\}

\author{Richard Killam}
\title{Why Python?}

%\setbeamercovered{transparent} 

\setbeamertemplate{navigation symbols}{}
%\setbeamertemplate{note page}[plain]
%\setbeameroption{show notes on second screen=right}

\addtobeamertemplate{frametitle}{}{
\begin{tikzpicture}[remember picture,overlay]
\node[anchor=north east,yshift=0pt] at (current page.north east) {\includegraphics[height=0.8cm]{python-logo.png}};
\end{tikzpicture}}

%\date{} 
%\institute{} 
%\logo{} 
%\subject{} 

\begin{document}
	\centering

	\begin{frame}
		\titlepage
	\end{frame}

	\begin{frame}{Outline}
		\begin{multicols}{2}
			\tableofcontents
		\end{multicols}
	\end{frame}

	\section{Motivation}
		\begin{frame}{Python Success Stories}
			\begin{itemize}
				\item Civ IV   
				\item Eve
    				\item Dropbox
    				\item Google
    				\item IceCube
    				\item LucasFilm
    				\item NASA
    				\item Reddit
    				\item Ubuntu
			\end{itemize}
		\end{frame}
		
		\begin{frame}{Biggest Motivator}
			\pause
    			\fontsize{30pt}{20pt}\selectfont
    			It's Free! \\
    			\emptyline
    			\& \\
    			\emptyline
    			Open Source!
		\end{frame}
	
	\section{Community}
	\subsection{PEP 8}
		\begin{frame}[fragile]{PEP 8 Rules}
			\begin{itemize}
				\item snake\_case\_everything
				\item except CapitalCamelCase class names
				\item \_private\_variable
				\item \_\_hidden\_private\_variable (Can't be seen by getattr)
				\item avoid\_keyword\_conflict\_
				\item \_\_magic\_functions\_\_ (Not as magical as advertised)
				\item Indentation, use spaces in multiples of 4 (4, 8, 12, ...)
				\item And many others (it's a really long list)
			\end{itemize}
		\end{frame}
		
		\begin{frame}[fragile]{PEP 8 Results}
			PHP v.s. Python
			\emptyline
			\emptyline
			
			\begin{itemize}
				\item<1-> strstr(string, substring) \uncover<2->{v.s. string.index(substring)}

				\item<3-> str\_replace(substring, replace\_with, string) \uncover<4->{v.s. string.replace(substring, replace\_with)}
							
				\item<5-> htmlspecialchars\_decode
				\item<5-> get\_html\_translation\_table
			\end{itemize}
		\end{frame}
		
	\subsection{Libraries}
		\begin{frame}{Batteries}
			\begin{itemize}
				\item Over 500 built-in libraries
				\item High performance wrappers of C libraries
				\item PIP: \textbf{P}ip \textbf{I}nstalls \textbf{P}ython has over 100 easy to install packages
				\item Github has 102,178 repositories written in Python, for Python
				\begin{itemize}
					\item[] Github user vinta compiled a list of awesome python packages: \url{https://github.com/vinta/awesome-python}
				\end{itemize}
			\end{itemize}
		\end{frame}
		
	\section{Simple Syntax}
	\subsection{Explicit}
		\begin{frame}[fragile]{Whitespace}
			What does this do?
			\emptyline
			\emptyline
			\emptyline

			\begin{minipage}{0.45\linewidth}
    				C-style (Java with printf):
			\end{minipage}	
			\begin{minipage}{0.45\linewidth}
    				Python:	
			\end{minipage}	
			
			\begin{minipage}{0.45\linewidth}
				\begin{lstlisting}[language=Java, basicstyle=\small]
					int s = 0;
					for(int i = 0; i < 10; ++i)
					  printf("%d + %d\n", s, i);
					  s += i;
				\end{lstlisting}
			\end{minipage}			
			\pause 
			\begin{minipage}{0.45\linewidth}
				\begin{lstlisting}[basicstyle=\small]
					s = 0;
					for i in xrange(10):
					  print("%d + %d\n" % (s, i))
					  s += i
				\end{lstlisting}
			\end{minipage}
		\end{frame}
		
		\begin{frame}[fragile]{\textbf{self}}
			What does this do?
			\emptyline
			\emptyline
			
			\begin{minipage}{0.45\linewidth}
    				C-style (Java with printf):
			\end{minipage}	
			\begin{minipage}{0.45\linewidth}
    				Python:	
			\end{minipage}				

			\begin{minipage}{0.45\linewidth}
				\begin{lstlisting}[language=Java, basicstyle=\small]
					class Test {
					    private int x;
					    public Test(int x) {
					    	    this.x = 2;
					    	    printf("%d\n", x);
					    	}
					    	public void printX() {
					    	    printf("%d\n", x);
					    	}
					 }
				\end{lstlisting}
			\end{minipage}			
			\pause 
			\begin{minipage}{0.45\linewidth}
				\begin{lstlisting}[basicstyle=\small]
					class Test(object):

					    def __init__(self, x):
					        self.x = 2
					    	    print(x)
					    	
					    	def print_x(self):
					    	    print(self.x)
				\end{lstlisting}
			\end{minipage}
		\end{frame}
		
	\subsection{Readable}
		\begin{frame}[fragile]{Looks like \textbf{Math}, Reads like \textbf{English}}
			\begin{itemize}
				\item<1-> (In)equality Chaining:
				\begin{itemize}
					\item $1 < x < y < z < 10$
					\item $x = y = z = 2$
				\end{itemize}
				
				\item<2-> 
				\begin{lstlisting}[language[Java]]
					for(int i = 0; i < my_list.length; ++i)
				\end{lstlisting} 
				$\qquad$ v.s.
				\begin{lstlisting}
				for variable in my_list
				\end{lstlisting}
				
				\item<3-> 
				\begin{lstlisting}
					if x in my_list
				\end{lstlisting}
				\uncover<3->{Still O(n) but incredibly well optimized}
			\end{itemize}
		\end{frame}
		
	\addtocontents{toc}{\newpage}
		
	\section{Built-in Data Structures}
	\subsection{list}
		\begin{frame}[fragile]{List Creation}
			\begin{itemize}
				\item<1-> List multiplication (works for strings too)
				\begin{lstlisting}
					my_list = [[1] * 2] * 2 (*@$\Rightarrow$@*) [[1, 1], 
					                               [1, 1]]
				\end{lstlisting}
				
				\item<2-> List comprehension
				\begin{lstlisting}
					[2**i for i in xrange(5)] (*@$\Rightarrow$@*) [1, 2, 4, 8, 16]
				\end{lstlisting}
			\end{itemize}
		\end{frame}
		
		\begin{frame}[fragile]{List comprehension v.s. filter, lambda, map, reduce}
			\begin{itemize}
			\item<1->
			\begin{lstlisting}[language=Bash]
			python -mtimeit -s'l=range(10)' 'map(lambda x:x+2,l)'
			    100000 loops, best of 3: 4.24 usec per loop

			python -mtimeit -s'l=range(10)' '[x+2 for x in l]'
			    100000 loops, best of 3: 2.32 usec per loop
			\end{lstlisting}

			\item<2-> Comprehensions more math like: set notation = $\{x + 2 \mid \forall x \epsilon l\}$

			\item<2-> More consistent with english: \\
				\centering{"Map item + 2 for all items in l"} \\
				\centering{v.s.} \\
				\centering{"I want a list of all of the items in l plus 2"}
			\end{itemize}
		\end{frame}
		
		\begin{frame}[fragile]{List Indexing}
			\begin{itemize}
				\item<1-> my\_list[my\_list.length - 1] \uncover<2->{v.s. my\_list[-1]}
				
				\item<3-> 
				\begin{lstlisting}[language=Java]
				for(int i = 1; i < my_list.length; i += 2)
				\end{lstlisting}
				\uncover<4->{v.s. my\_list[1::2]}
				
				\item<5->
				\begin{itemize}
                		\item[] range(5)[1:]    $\:\:\:\:\:\Rightarrow$ [1, 2, 3, 4]
                		\item[] range(5)[:-1]   $\:\:\:\:\Rightarrow$ [0, 1, 2, 3]
                		\item[] range(5)[1:3]   $\:\:\:\Rightarrow$ [1, 2]
                		\item[] range(10)[::2]  $\:\:\Rightarrow$ [0, 2, 4, 6, 8]
                		\item[] range(10)[1::2] $\Rightarrow$ [1, 3, 5, 7, 9]
				\end{itemize}
			\end{itemize}
		\end{frame}
		
	\subsection{set}
		\begin{frame}[fragile]{Sets}
			\begin{itemize}
				\item<1-> Remove duplicates:
				\begin{lstlisting}
					set([1, 2, 3, 4, 1, 2, 3, 4]) (*@$\Rightarrow$@*) {1, 2, 3, 4}
				\end{lstlisting}
				
				\item<2->
				\begin{lstlisting}
					if x in my_set:
				\end{lstlisting}
				\begin{itemize}
					\item<2->[Avg:] \textbf{O(1)}
					\item<2->[Worst:] O(n)
				\end{itemize}
			\end{itemize}
		\end{frame}
		
		\begin{frame}[fragile]{Set Operations}
			\begin{itemize}
				\item set1 - set2 == set1.difference(set2) \\ O(len(set1))
				\pause
					
				\item set1 \& set2 == set1.intersection(set2) \\ Avg: O(min(len(set1, set2)))
				\pause				
								
				\item set1 $|$ set2  == set1.union(set2) \\ O(len(set1)+len(set2))
				\pause
						
				\item set1 \^{} set2 == set1.symmetric\_difference(set2) \\ Avg: O(len(set1))
			\end{itemize}
		\end{frame}
		
	\subsection{dict}
		\begin{frame}[fragile]{Dictionaries}
			\begin{itemize}
				\item Built-in hash map
				
				\item O(1) lookups:
				\begin{lstlisting}
				    my_dict.get(key)  # None if key does not exist
				\end{lstlisting}
				
				\item dict comprehension:
				\begin{lstlisting}
					{obj.name: obj.data for obj in my_objs}
				\end{lstlisting}
				
				\item Easy looping:
				\begin{lstlisting}
					for key, val in my_dict.items():
					    print('{}: {}'.format(key, val))
				\end{lstlisting}
			\end{itemize}
		\end{frame}
		
	\section{Language Constructs}
	\subsection{Context Managers}
		\begin{frame}[fragile]{What are Context Managers?}
			Handles allocation and release of a resource \\
			\emptyline
			\emptyline
			
			\begin{minipage}{0.45\linewidth}
				This:
			\end{minipage}
			\begin{minipage}{0.45\linewidth}
				Becomes:
			\end{minipage}
			
			\begin{minipage}{0.45\linewidth}
				% FIXME: have a better representation of what with open does
				\begin{lstlisting}
				f = None
				try:
				    f = open('f.txt', 'r')
				    # Do stuff...
				finally:
				    if f is not None:
				        f.close()
				\end{lstlisting}
			\end{minipage}
			\pause
			\begin{minipage}{0.45\linewidth}
				\begin{lstlisting}
				with open('f.txt', 'r') as f:
				    # Do stuff...
				\end{lstlisting}
			\end{minipage}
		\end{frame}
		
		\begin{frame}[fragile]{Why use Context Managers?}
			\begin{itemize}
				\item Avoid verbose repeat code
				\item Ensure release is handled properly
				\item Variable scope retention
			\end{itemize}
		\end{frame}
		
		\begin{frame}[fragile]{Context Manager Uses}
			\begin{itemize}
				\item Ensure successful db transaction before commit
				\item Holding some I/O
				\item Locking a thread
				\item Opening a file
			\end{itemize}
		\end{frame}
	
	\subsection{Decorators}
		\begin{frame}[fragile]{What are Decorators?}
			Classes or higher order functions that wrap a given function or class \\
			\emptyline
			\emptyline
			
			\begin{minipage}{0.45\linewidth}
				This:
			\end{minipage}
			\begin{minipage}{0.45\linewidth}
				Becomes:
			\end{minipage}
			
			\begin{minipage}{0.45\linewidth}
				\begin{lstlisting}
					def my_f(...):  
					    Do stuff...         
					    ret = f(...)        
					    Do more stuff...    
					    return ret
				\end{lstlisting}
			\end{minipage}
			\pause
			\begin{minipage}{0.45\linewidth}
				\begin{lstlisting}
					@my_decorator
					def f(...):
					    ...
				\end{lstlisting}
			\end{minipage}
		\end{frame}
		
		\begin{frame}[fragile]{Why use Decorators?}
			\begin{itemize}
				\item Avoid verbose repeat code
				\item Closures allow for state retention:
				\begin{itemize}
					\item[] aggregation, memoization, etc
				\end{itemize}
			\end{itemize}
		\end{frame}
		
		\begin{frame}[fragile]{Decorator Uses}
			\begin{itemize}
				\item Argument/return checking
				\item Function timeout
				\item Logging (decorate class)
				\item Memoization
				\item Thunkifying (Parallelizing)
			\end{itemize}
		\end{frame}
	
	\subsection{Generators}
		\begin{frame}[fragile]{What are Generators?}
			An easy way to support iterations \\
			
			\begin{minipage}{0.45\linewidth}
				This:
			\end{minipage}
			\begin{minipage}{0.45\linewidth}
				Becomes:
			\end{minipage}
			
			\begin{minipage}{0.45\linewidth}
				\begin{lstlisting}
					class Test(object):
					    def __init__(sf,s,e):
					        sf.c = s
					        sf.e = e 
					    def __iter__(sf):
					        return sf
					    def next(sf):
					        if sf.c>=sf.e:
					            raise StopIter
					        r = sf.c
					        sf.c += 1
					        return r
				\end{lstlisting}
			\end{minipage}
			\pause
			\begin{minipage}{0.45\linewidth}
				\begin{lstlisting}
					def test(start, end):
					    while start < end:
					        yield start     
					        start += 1
				\end{lstlisting}
			\end{minipage}
		\end{frame}
		
		\begin{frame}[fragile]{Why use Generators?}
			\begin{itemize}
				\item Lazy evaluation
				\item Less memory usage
			\end{itemize}
		\end{frame}
		
		\begin{frame}[fragile]{Generator Uses}
			\begin{itemize}
				\item Co-routines (producer/consumer) using two-way generators
				\item Interpolations/regressions (unknown number of iterations)
				\item Process text files
			\end{itemize}
		\end{frame}
		
	\section{Python and Scientific Computing}
		\begin{frame}{IPython}
			Featureful Python REPL
			\emptyline
			\emptyline
			\emptyline
			\pause
			
			\begin{itemize}
				\item Explore objects (my\_obj?)
				\item Magic functions (debug, edit, run, timeit, ...)
				\item Multi shell support (bash, javascript, latex, perl, pypy, ruby)
				\item Notebooks allow for fast and easy sharing of code and data 
			\end{itemize}
		\end{frame}
		
		\begin{frame}{Scientific Computing Libraries}
			\begin{itemize}
				\item \textbf{numpy}: Flexible array structures for fast mathematical operations
				\item \textbf{pandas}: A powerful data analysis and manipulation library
				
				\item \textbf{scipy}: 32 subpackages for scientific and mathematical operations				
				\item \textbf{scikit-learn (sklearn)}: Easy to use machine learning library
				\item \textbf{nltk}: Massive natural language processing library
				\item \textbf{cv/cv2}: Python wrappers for the fast and powerful OpenCV computer vision library
				\item \textbf{matplotlib}: Easy to use plotting library
				\item \textbf{pickle}: Object serializing
			\end{itemize}			 
		\end{frame}
		
		\begin{frame}[fragile]{Other Features (Python Buzzwords)}
			\begin{itemize}
				\item \textbf{Virtualenv}:
				\begin{itemize}
					\item[] Encapsulate python projects and their dependencies
				
				\end{itemize}
				\item \textbf{MicroPython}: Python for micro controllers 
				\item \textbf{Namespaces}: Built-in way to encapsulate your modules
				\item \textbf{Cython and PyPy}: Python speed boosts
				\item \textbf{CPython, Jython, IronPython, RPython, pyjs}: 
				\begin{itemize}
					\item[] Multi-language support (C, Java, .NET, R, JavaScript, ...)
				\end{itemize}
				\item \textbf{Brython}: Python in the browser
				\item \textbf{Django, Flask, Bottle}: Python on the server
			\end{itemize}
		\end{frame}
		
		\begin{frame}{
			The Zen of Python, by Tim Peters: import this}
    			\fontsize{10pt}{10pt}\selectfont
			
			\begin{minipage}{0.45\linewidth}
			Beautiful is better than ugly.\\
			Explicit is better than implicit.\\
			Simple is better than complex.\\
			Complex is better than complicated.\\
			Flat is better than nested.\\
			Sparse is better than dense.\\
			Readability counts.\\
			Special cases aren't special enough to break the rules.\\
			Although practicality beats purity.\\
			Errors should never pass silently.\\
			Unless explicitly silenced.\\
			In the face of ambiguity, refuse the temptation to guess.\\
			\end{minipage}
			\hspace{0.5cm}
			\begin{minipage}{0.45\linewidth}
			There should be one-- and preferably only one --obvious way to do it.\\
			Although that way may not be obvious at first unless you're Dutch.\\
			Now is better than never.\\
			Although never is often better than *right* now.\\
			If the implementation is hard to explain, it's a bad idea.\\
			If the implementation is easy to explain, it may be a good idea.\\
			Namespaces are one honking great idea -- let's do more of those!\\
			\end{minipage}
		\end{frame}
		
\end{document}