\documentclass[11pt]{beamer}
\usetheme{Copenhagen}
\usecolortheme{dolphin}

\usepackage[utf8]{inputenc}
\usepackage{amsmath}
\usepackage{amsfonts}
\usepackage{amssymb}
\usepackage{array}
\usepackage{caption}
\usepackage{datetime}
\usepackage{fancyhdr}
\usepackage{gensymb}
\usepackage{graphicx}
\usepackage{lastpage}
\usepackage{multicol}
\usepackage{multirow}
\usepackage{pgfpages}
\usepackage{setspace}
\usepackage{tikz}

% Code snippets
\usepackage{listings}
\usepackage{color}

\definecolor{dkgreen}{rgb}{0,0.6,0}
\definecolor{gray}{rgb}{0.5,0.5,0.5}
\definecolor{mauve}{rgb}{0.58,0,0.82}

\lstset{
  basicstyle={\small\ttfamily},
  breakatwhitespace=true,
  breaklines=true,
  columns=flexible,
  commentstyle=\color{dkgreen},
  escapeinside={(*@}{@*)},
  keywordstyle=\color{blue},
  numbers=left,
  numberstyle=\color{gray},
  showstringspaces=false,
  stringstyle=\color{mauve},
  tabsize=2,
  language=Python,
  morekeywords={as, None, with, yield, True, False},
  emph={self},
  emphstyle=\color{red}
}

\newcommand{\code}[1]{
\begin{lstlisting}
#1
\end{lstlisting}}

\newcommand{\emptyline}{$ $\\}

\author{Richard Killam}
\title{Introduction to Cython - Week 1}

%\setbeamercovered{transparent} 

\setbeamertemplate{navigation symbols}{}
%\setbeamertemplate{note page}[plain]
%\setbeameroption{show notes on second screen=right}

\addtobeamertemplate{frametitle}{}{
\begin{tikzpicture}[remember picture,overlay]
\node[anchor=north east,yshift=0pt] at (current page.north east) {\includegraphics[height=0.8cm]{Cython-logo.png}};
\end{tikzpicture}}

%\date{} 
%\institute{} 
%\logo{} 
%\subject{} 

\begin{document}
\centering

\begin{frame}
	\titlepage
\end{frame}

\begin{frame}{Outline}
	\begin{multicols}{2}
		\tableofcontents
	\end{multicols}
\end{frame}

\section{Motivation}
\begin{frame}{Motivation}
	Sum the numbers from 1 to 100,000,000

	\begin{description}[style=multiline,topsep=10pt]
		\item[Python] 12.898s
		\item[Compiled Python] 12.177s
		\item[Cython] 0.432s
		\item[C] 0.406s
	\end{description}
\end{frame}

\section{I/O}
\begin{frame}[fragile]{Hello World}
	\begin{minipage}{0.45\linewidth}
		hello\_world.py
		\begin{lstlisting}
			print("Hello World")
		\end{lstlisting}
	\end{minipage}
	\pause
	\begin{minipage}{0.45\linewidth}
		python hello\_world.py
		\emptyline
		$>>$ Hello World
	\end{minipage}
\end{frame}

\begin{frame}[fragile]{My name is...}
	my\_name\_is.py
	\begin{lstlisting}
		name = raw_input("Name > ")
		print("Hello World, my name is {NAME}".format(NAME=name))
	\end{lstlisting}

	\pause
	
	python my\_name\_is.py
	
	$>>$ Name $>$ 
	
	$>>$ Name $>$ Richard
	
	$>>$ Hello World, my name is Richard
	
\end{frame}

\subsection{Exercise}
\begin{frame}[fragile]{Exercise - My full name is...}
	\pause
	my\_full\_name\_is.py
	\begin{lstlisting}
		first_name = raw_input("First name > ")
		last_name = raw_input("Last name > ")
		
		print("Hello World, my name is {FIRST} {LAST}".format(
		    FIRST=first_name, 
		    LAST=last_name
		))
	\end{lstlisting}
\end{frame}

\begin{frame}[fragile]{Getting numbers in}
	name\_and\_age.py
	\begin{lstlisting}
		name = raw_input("Name > ")
		year_of_birth = int(raw_input("Year of Birth > "))
		
		print("{NAME} is {YEARS} old.".format(
		    NAME=name,
		    YEARS=2015 - year_of_birth		
		))
	\end{lstlisting}
\end{frame}

\section{Conditionals}
\subsection{if, else, and elif}
\begin{frame}[fragile]{if, else}
	True\_False.py
	\begin{lstlisting}
		if True:
		    print("The conditional was True")
		
		else:
		    print("The conditional was False")
	\end{lstlisting}
	\emptyline
	\emptyline
	\pause
	Note the indentation.
\end{frame}

\begin{frame}[fragile]{if, else with input}
	even\_or\_odd.py
	\begin{lstlisting}
		num = int(raw_input("Number > "))
		
		if num % 2 == 0:
		    print("{NUM} is even".format(NUM=num))
		
		else:
		    print("{NUM} is odd".format(NUM=num))
	\end{lstlisting}
\end{frame}

\begin{frame}[fragile]{elif}
	a\_or\_b.py
	\begin{lstlisting}
		choice = raw_input("Choose a or b ")
		
		if choice == "a":
		    print("You chose a")
		
		elif choice == "b":
		    print("You chose b")
		
		else:
		    print("You did not follow instructions...")
	\end{lstlisting}
\end{frame}

\subsection{While}
\begin{frame}[fragile]{While + and}
	force\_a\_or\_b.py
	\begin{lstlisting}
		choice = ""
		
		while choice != "a" and choice != "b":
			choice = raw_input("Choose a or b >")
		
		if choice == "a":
		    print("You chose a")
		
		elif choice == "b":
		    print("You chose b")
		
		else:
		    # Dead code
		    print("How did you get here?!?!")
	\end{lstlisting}
\end{frame}

\begin{frame}[fragile]{While + or + not}
	force\_a\_or\_b.py
	\begin{lstlisting}
		choice = ""
		
		while not (choice == "a" or choice == "b"):
			choice = raw_input("Choose a or b >")
		
		if choice == "a":
		    print("You chose a")
		
		elif choice == "b":
		    print("You chose b")
		
		else:
		    # Dead code
		    print("How did you get here?!?!")
	\end{lstlisting}
\end{frame}

\section{Iteration}
\subsection{Arrays}
\begin{frame}[fragile]{Array declaration}
	array\_declaration.py
	\begin{lstlisting}
		a = [1, 2, 3, 4, 5]
		print(a)  # >> [1, 2, 3, 4, 5]
		
		a = range(1, 6)
		print(a)  # >> [1, 2, 3, 4, 5]
		
		a = range(5)
		print(a)  # >> [0, 1, 2, 3, 4]
	\end{lstlisting}
\end{frame}

\begin{frame}[fragile]{Array indexing}
	array\_indexing.py
	\begin{lstlisting}
		a = range(5)
		
		print(a)  # >> [0, 1, 2, 3, 4]
		print(a[0])  # >> 0
		
		print(a[len(a) - 1])  # >> 4
		print(a[-1])  # >> 4
		
		print(a[len(a) - 2])  # >> 3
		print(a[-2])  # >> 3
	\end{lstlisting}
\end{frame}

\subsection{for loop}
\begin{frame}[fragile]{C-Style for loop}
	c\_style\_for\_loop.py
	\begin{lstlisting}
		a = range(5, 10)
		
		for i in range(len(a)):
		    print("a[{I}] = {AI}".format(I=i, AI=a[i]))
	\end{lstlisting}
\end{frame}

\begin{frame}[fragile]{Python-Style for loop}
	python\_style\_for\_loop.py
	\begin{lstlisting}
		a = range(5, 10)
		
		for ai in a:
		    print("AI = {AI}".format(AI=a[i])) 
		(*@\pause@*)
		for i, ai in enumerate(a):
		    print("a[{I}] = {AI}".format(I=i, AI=ai))
	\end{lstlisting}
\end{frame}

\subsection{Exercise}
\begin{frame}[fragile]{Exercise - Sum the Numbers from 1 to n}
	sum\_nums.py	
	\pause
	\begin{lstlisting}
		n = int(raw_input("n (-1 to quit) > "))
		while n > 1:
		    s = 0		 
		    
		    # n + 1 because range goes to n - 1   
		    for i in range(1, n+1):
		    	    s += i
		    	
		    	print("n = {N} -> {S}".format(N=n, S=s))
		    
		    n = int(raw_input("n (-1 to quit) > "))
	\end{lstlisting}
\end{frame}

\section{Functions}
\begin{frame}[fragile]{def}
	sum\_nums\_func.py
	\begin{lstlisting}
		def sum_nums(n):
		    s = 0
		    
		    # n + 1 because range goes to n - 1
		    for i in range(1, n+1):
		        s += i
			
		    return s
		
		print("sum_nums({N}) => {SNN}".format(
		    N=10, 
		    SNN=sum_nums(10)
		))  # >> 55
	\end{lstlisting}
\end{frame}

\begin{frame}[fragile]{Default Arguments}
	func\_default\_args.py
	\begin{lstlisting}
		def default_args(a=1, b=3):
		    print("a was {A}; b was {B}".format(A=a, B=b))
		
		default_args()             # >> a was 1; b was 3
		default_args(7)            # >> a was 7; b was 3
		default_args(7, 11)        # >> a was 7; b was 11
		default_args(7, b=12)      # >> a was 7; b was 12
		default_args(a=8, b=12)    # >> a was 8; b was 12
		default_args(b=13, a=5)    # >> a was 5; b was 13
		default_args(a=10, 14)     # >> SyntaxError: non-keyword arg after keyword arg
	\end{lstlisting}
\end{frame}

\subsection{Exercise}
\begin{frame}[fragile]{Exercise - Sum the Numbers from low to high}
	sum\_range.py
	\pause
	\begin{lstlisting}
		def sum_range(low=1, high=10):
		    s = 0
		    
		    # high + 1 because range goes to high - 1
		    for i in range(low, high + 1):
		        s += i
			
		    return s
	\end{lstlisting}
\end{frame}

\section{Importing}
\subsection{From Python Path}
\begin{frame}[fragile]{Basics}
	basic\_import\_from\_python\_path.py
	\begin{lstlisting}
		# imports that math library and references it as math
		import math
		print(math.log(100, 10))  # >> 2.0  (log 100 base 10)
		
		# sys.argv allows for command line arguments
		import sys
		print(sys.argv)
	\end{lstlisting}
\end{frame}
		
\begin{frame}[fragile]{Different styles of importing}
	\begin{lstlisting}
		# imports the sparse module from the scipy library as scipy.sparse
		import scipy.sparse
		
		# imports the sparse module from the scipy library as sp_sparse
		import scipy.sparse as sp_sparse
		
		# imports the sparse module from the scipy library as sparse
		from scipy import sparse
	\end{lstlisting}
\end{frame}

\subsection{From Current Folder}
\begin{frame}[fragile]{From sum\_range.py}
	import\_sum\_range.py
	\begin{lstlisting}
		# This is the only style that I will use in the workshop
		import sum_range  # Note that there is no .py
		print(sum_range.sum_range(low=2, high=10))  # >> 54
		
		import sum_range as sr
		print(sr.sum_range(low=2, high=10))  # >> 54
		
		from sum_range import sum_range
		print(sum_range(low=2, high=10))  # >> 54
		
		from sum_range import sum_range as sr
		print(sr(low=2, high=10))  # >> 54
	\end{lstlisting}
\end{frame}

\section{Classes}
\begin{frame}[fragile]{Definition, Instantiation, and Usage}
	animal.py
	\begin{lstlisting}
		# Explicit inheritance from object class
		class Animal(object):
		    # Explicit (and necessary) passing of self object
		    def __init__(self, name_in, noise_in):
		        self.name = name_in  # (*@ $\not\equiv$ @*) name = name_in
		        self.noise = noise_in  # (*@ $\not\equiv$ @*) noise = noise_in
		
		dog = Animal("Rex", "woof")
		print("{DOG} makes a {NOISE} noise".format(
		    DOG=dog.name,
		    NOISE=dog.noise
		))
	\end{lstlisting}
\end{frame}

\begin{frame}[fragile]{Inheritance}
	animal.py
	\begin{lstlisting}
		class Dog(Animal):
		    def __init__(self, name):
		        # This gets better in Python 3
		        super(Dog, self).__init__(name, "woof")
		    
		    # NEED to accept self
		    def wag_tail(self):
		    	    print("{NAME} is happy".format(NAME=self.name))
		 
		 rex = Dog("Rex")
		 rex.wag_tail()  # DON'T need to explicitly pass self
	\end{lstlisting}
\end{frame}

\end{document}