\documentclass[11pt]{beamer}
\usetheme{Copenhagen}
\usecolortheme{dolphin}

\usepackage[utf8]{inputenc}
\usepackage{amsfonts}
\usepackage{amsmath}
\usepackage{amssymb}
\usepackage{array}
\usepackage{caption}
\usepackage{datetime}
\usepackage{fancyhdr}
\usepackage{gensymb}
\usepackage{graphicx}
\usepackage{lastpage}
\usepackage{multicol}
\usepackage{multirow}
\usepackage{pgfpages}
\usepackage{setspace}
\usepackage{tikz}

% Code snippets
\usepackage{listings}
\usepackage{color}

\definecolor{dkgreen}{rgb}{0,0.6,0}
\definecolor{gray}{rgb}{0.5,0.5,0.5}
\definecolor{mauve}{rgb}{0.58,0,0.82}

\lstset{
  basicstyle={\small\ttfamily},
  breakatwhitespace=true,
  breaklines=true,
  columns=flexible,
  commentstyle=\color{dkgreen},
  escapeinside={(*@}{@*)},
  keywordstyle=\color{blue},
  numbers=left,
  numberstyle=\color{gray},
  showstringspaces=false,
  stringstyle=\color{mauve},
  tabsize=2,
  language=Python,
  morekeywords={as, None, with, yield, True, False, self, cdef},
  emph={cpdef, cimport},
  emphstyle=\color{red}
}

\newcommand{\code}[1]{
\begin{lstlisting}
#1
\end{lstlisting}}

\newcommand{\emptyline}{$ $\\}

\author{Richard Killam}
\title{Introduction to Cython - Week 3 \\ (Alternate: Python Development)}

%\setbeamercovered{transparent}

\setbeamertemplate{navigation symbols}{}
%\setbeamertemplate{note page}[plain]
%\setbeameroption{show notes on second screen=right}

\addtobeamertemplate{frametitle}{}{
\begin{tikzpicture}[remember picture,overlay]
\node[anchor=north east,yshift=0pt] at (current page.north east) {\includegraphics[height=0.8cm]{Cython-logo.png}};
\end{tikzpicture}}

%\date{}
%\institute{}
%\logo{}
%\subject{}

\begin{document}

\begin{frame}
    \titlepage
\end{frame}

\begin{frame}{Outline}
    \tableofcontents
\end{frame}

\section{Basic Data Structures}
\subsection{list}
\begin{frame}[fragile]{List Creation}
    \begin{itemize}
        \item<1-> List multiplication (works for strings too)
        \begin{lstlisting}
my_list = [[1] * 2] * 2 (*@$\Rightarrow$@*) [[1, 1],
                               [1, 1]]
        \end{lstlisting}

        \item<2-> List comprehension
        \begin{lstlisting}
[2**i for i in xrange(5)] (*@$\Rightarrow$@*) [1, 2, 4, 8, 16]
        \end{lstlisting}
    \end{itemize}
\end{frame}

\begin{frame}[fragile]{List Indexing}
    \begin{itemize}
        \item<1-> my\_list[my\_list.length - 1] \uncover<2->{v.s. my\_list[-1]}

        \item<3-> $ $ \\
        \begin{lstlisting}[language=Java]
for(int i = 1; i < my_list.length; i += 2)
        \end{lstlisting}
        \uncover<4->{v.s. my\_list[1::2]}

        \item<5->
        \begin{itemize}
                \item[] range(5)[1:]    $\:\:\:\:\:\Rightarrow$ [1, 2, 3, 4]
                \item[] range(5)[:-1]   $\:\:\:\:\Rightarrow$ [0, 1, 2, 3]
                \item[] range(5)[1:3]   $\:\:\:\Rightarrow$ [1, 2]
                \item[] range(10)[::2]  $\:\:\Rightarrow$ [0, 2, 4, 6, 8]
                \item[] range(10)[1::2] $\Rightarrow$ [1, 3, 5, 7, 9]
        \end{itemize}
    \end{itemize}
\end{frame}

\subsection{set}
\begin{frame}[fragile]{Sets}
    \begin{itemize}
        \item<1-> Remove duplicates:
        \begin{lstlisting}
set([1, 2, 3, 4, 1, 2, 3, 4]) (*@$\Rightarrow$@*) {1, 2, 3, 4}
        \end{lstlisting}

        \item<2-> $ $ \\
        \begin{lstlisting}
if x in my_set:
        \end{lstlisting}
        \begin{itemize}
            \item<2->[Avg:] \textbf{O(1)}
            \item<2->[Worst:] O(n)
        \end{itemize}
    \end{itemize}
\end{frame}

\begin{frame}[fragile]{Set Operations}
    \begin{itemize}
        \item set1 - set2 == set1.difference(set2) \\ O(len(set1))
        \pause

        \item set1 \& set2 == set1.intersection(set2) \\ Avg: O(min(len(set1, set2)))
        \pause

        \item set1 $|$ set2  == set1.union(set2) \\ O(len(set1)+len(set2))
        \pause

        \item set1 \^{} set2 == set1.symmetric\_difference(set2) \\ Avg: O(len(set1))
    \end{itemize}
\end{frame}

\subsection{dict}
\begin{frame}[fragile]{Dictionaries}
    \begin{itemize}
        \item Built-in hash map

        \item O(1) lookups:
        \begin{lstlisting}
my_dict.get(key)  # None if key does not exist
        \end{lstlisting}

        \item dict comprehension:
        \begin{lstlisting}
{obj.name: obj.data for obj in my_objs}
        \end{lstlisting}

        \item Easy looping:
        \begin{lstlisting}
for key, val in my_dict.items():
    print('{}: {}'.format(key, val))
        \end{lstlisting}
    \end{itemize}
\end{frame}

\section{Language Constructs}
\subsection{Context Managers}
\begin{frame}[fragile]{What are Context Managers?}
    Handles allocation and release of a resource \\
    \emptyline
    \emptyline

    \begin{minipage}{0.45\linewidth}
        This:
    \end{minipage}
    \begin{minipage}{0.45\linewidth}
        Becomes:
    \end{minipage}

    \begin{minipage}{0.45\linewidth}
        % FIXME: have a better representation of what with open does
        \begin{lstlisting}
f = None
try:
    f = open('f.txt', 'r')
    # Do stuff...
finally:
    if f is not None:
        f.close()
        \end{lstlisting}
    \end{minipage}
    \pause
    \begin{minipage}{0.45\linewidth}
        \begin{lstlisting}
with open('f.txt', 'r') as f:
    # Do stuff...
        \end{lstlisting}
    \end{minipage}
\end{frame}

\begin{frame}[fragile]{Why use Context Managers?}
    \begin{itemize}
        \item Avoid verbose repeat code
        \item Ensure release is handled properly
        \item Variable scope retention
    \end{itemize}
\end{frame}

\begin{frame}[fragile]{Context Manager Uses}
    \begin{itemize}
        \item Ensure successful db transaction before commit
        \item Holding some I/O
        \item Locking a thread
        \item Opening a file
    \end{itemize}
\end{frame}

\subsection{Decorators}
\begin{frame}[fragile]{What are Decorators?}
    Classes or higher order functions that wrap a given function or class \\
    \emptyline
    \emptyline

    \begin{minipage}{0.45\linewidth}
        This:
    \end{minipage}
    \begin{minipage}{0.45\linewidth}
        Becomes:
    \end{minipage}

    \begin{minipage}{0.45\linewidth}
        \begin{lstlisting}
def my_f(...):
    Do stuff...
    ret = f(...)
    Do more stuff...
    return ret
        \end{lstlisting}
    \end{minipage}
    \pause
    \begin{minipage}{0.45\linewidth}
        \begin{lstlisting}
@my_decorator
def f(...):
    ...
        \end{lstlisting}
    \end{minipage}
\end{frame}

\begin{frame}[fragile]{Why use Decorators?}
    \begin{itemize}
        \item Avoid verbose repeat code
        \item Closures allow for state retention:
        \begin{itemize}
            \item[] aggregation, memoization, etc
        \end{itemize}
    \end{itemize}
\end{frame}

\begin{frame}[fragile]{Decorator Uses}
    \begin{itemize}
        \item Argument/return checking
        \item Function timeout
        \item Logging (decorate class)
        \item Memoization
        \item Thunkifying (Parallelizing)
    \end{itemize}
\end{frame}

\subsection{Generators}
\begin{frame}[fragile]{What are Generators?}
    An easy way to support iterations \\

    \begin{minipage}{0.45\linewidth}
        This:
    \end{minipage}
    \begin{minipage}{0.45\linewidth}
        Becomes:
    \end{minipage}

    \begin{minipage}{0.45\linewidth}
        \begin{lstlisting}
class Test(object):
    def __init__(sf,s,e):
        sf.c = s
        sf.e = e
    def __iter__(sf):
        return sf
    def next(sf):
        if sf.c>=sf.e:
            raise StopIter
        r = sf.c
        sf.c += 1
        return r
        \end{lstlisting}
    \end{minipage}
    \pause
    \begin{minipage}{0.45\linewidth}
        \begin{lstlisting}
def test(start, end):
    while start < end:
        yield start
        start += 1
        \end{lstlisting}
    \end{minipage}
\end{frame}

\begin{frame}[fragile]{Why use Generators?}
    \begin{itemize}
        \item Lazy evaluation
        \item Less memory usage
    \end{itemize}
\end{frame}

\begin{frame}[fragile]{Generator Uses}
    \begin{itemize}
        \item Co-routines (producer/consumer) using two-way generators
        \item Interpolations/regressions (unknown number of iterations)
        \item Process text files
    \end{itemize}
\end{frame}

\end{document}

