\documentclass[11pt]{beamer}
\usetheme{Copenhagen}
\usecolortheme{dolphin}

\usepackage[utf8]{inputenc}
\usepackage{amsmath}
\usepackage{amsfonts}
\usepackage{amssymb}
\usepackage{array}
\usepackage{caption}
\usepackage{datetime}
\usepackage{fancyhdr}
\usepackage{gensymb}
\usepackage{graphicx}
\usepackage{lastpage}
\usepackage{multicol}
\usepackage{multirow}
\usepackage{pgfpages}
\usepackage{setspace}
\usepackage{tikz}

% Code snippets
\usepackage{listings}
\usepackage{color}

\definecolor{dkgreen}{rgb}{0,0.6,0}
\definecolor{gray}{rgb}{0.5,0.5,0.5}
\definecolor{mauve}{rgb}{0.58,0,0.82}

\lstset{
  basicstyle={\small\ttfamily},
  breakatwhitespace=true,
  breaklines=true,
  columns=flexible,
  commentstyle=\color{dkgreen},
  escapeinside={(*@}{@*)},
  keywordstyle=\color{blue},
  numbers=left,
  numberstyle=\color{gray},
  showstringspaces=false,
  stringstyle=\color{mauve},
  tabsize=2,
  language=Python,
  morekeywords={as, None, with, yield, True, False},
  emph={self, cdef, cpdef},
  emphstyle=\color{red}
}

\newcommand{\code}[1]{
\begin{lstlisting}
#1
\end{lstlisting}}

\newcommand{\emptyline}{$ $\\}

\author{Richard Killam}
\title{Introduction to Cython - Week 2}

%\setbeamercovered{transparent} 

\setbeamertemplate{navigation symbols}{}
%\setbeamertemplate{note page}[plain]
%\setbeameroption{show notes on second screen=right}

\addtobeamertemplate{frametitle}{}{
\begin{tikzpicture}[remember picture,overlay]
\node[anchor=north east,yshift=0pt] at (current page.north east) {\includegraphics[height=0.8cm]{Cython-logo.png}};
\end{tikzpicture}}

%\date{} 
%\institute{} 
%\logo{} 
%\subject{} 

\begin{document}
\centering

\begin{frame}
	\titlepage
\end{frame}

\begin{frame}{Outline}
	\begin{multicols}{2}
		\tableofcontents
	\end{multicols}
\end{frame}

\section{Using Cython Files}
\begin{frame}[fragile]{Method Overview}
	There are 3 ways to use Cython files:
	\begin{enumerate}[<+->]
		\item \textbf{Direct Import:} import the code without \underline{explicitly} compiling
		\item \textbf{Compiled Import:} explicitly compile the code, then import
		\item \textbf{Compiled Executable:} explicitly compile the code and then run it directly
	\end{enumerate}
	
	\pause[\thebeamerpauses]
	hello.pyx
	\lstinputlisting{../Examples/CythonHelloWorld/DirectImport/hello.pyx}
\end{frame}

\begin{frame}[fragile]{Direct Import}
	\begin{lstlisting}[language=Bash]
		cd Examples/CythonHelloWorld/DirectImport
		ls
		python run_hello.py
	\end{lstlisting}
	\pause
	run\_hello.py
	\lstinputlisting{../Examples/CythonHelloWorld/DirectImport/run_hello.py}
\end{frame}

\begin{frame}[fragile]{Compiled Import}
	\begin{lstlisting}[language=Bash]
		cd Examples/CythonHelloWorld/CompiledImport
		ls
	\end{lstlisting}
	
	\begin{description}
		\item[setup.py] Helper script compiles the given .pyx files into C libraries (.so files)
		\lstinputlisting{../Examples/CythonHelloWorld/CompiledImport/setup.py}
	\end{description}
\end{frame}

\begin{frame}[fragile]{Compiled Import cont.}
	\begin{lstlisting}[language=Bash]
		python run_hello.py (*@\pause@*) # ImportError (*@\pause@*)
		python setup.py build_ext --inplace
		ls  # Note hello.so
		python run_hello.py
	\end{lstlisting}
	
	run\_hello.py
	\lstinputlisting{../Examples/CythonHelloWorld/CompiledImport/run_hello.py}

	\pause
	Notice the speed difference between Direct and Compiled Importing.
\end{frame}

\begin{frame}[fragile]{Compiled Executable}
	\begin{lstlisting}[language=Bash]
		cd Examples/CythonHelloWorld/CompiledExecutable
		ls	
	\end{lstlisting}
	
	\begin{flushleft}
		\textbf{cython\_build.sh} Script I wrote to streamline the compilation process. 
		\begin{enumerate}
			\item Uses the Cython compiler to compile hello.pyx into hello.c
			\item Uses gcc to compile hello.c into an executable
		\end{enumerate}
	\end{flushleft}
\end{frame}

\begin{frame}[fragile]{Compiled Executable cont}
	\begin{lstlisting}[language=Bash]
		bash cython_build.sh hello.pyx
		./hello
	\end{lstlisting}
	
	\pause
	Open hello.c
	\pause
	\begin{lstlisting}[language=Bash]
		wc -l hello.c  # 1,626 lines!!!
	\end{lstlisting}
\end{frame}

\begin{frame}[fragile]{Method Summary}
	\begin{itemize}[<+->]
		\item \textbf{Direct Import} 
		\begin{itemize}[<.->]
			\item Slow start-up on each run
			\item Simple
			\item Good for development
		\end{itemize}
		
		\item \textbf{Compiled Import}
		\begin{itemize}[<.->]
			\item Fast on each run
			\item Added step in execution process
			\item Used for production / release
		\end{itemize}
		
		\item \textbf{Compiled Executable}
		\begin{itemize}[<.->]
			\item Complicated compilation process
			\item Could be used to develop a module
			\item Most used method for this workshop
		\end{itemize}
	\end{itemize}
\end{frame}

\subsection{Exercise}
\begin{frame}[fragile]{sum\_nums\_func.pyx}
	sum\_nums\_func.pyx
	\pause
	\begin{lstlisting}
		def sum_nums(n):
		    s = 0
		    for i in range(n+1):
		        s += i
		    return s
		
		import sys
		n = int(sys.argv[1])
		print(sum_nums(n))
	\end{lstlisting}
	\pause
	\begin{lstlisting}[language=Bash]
		time python sum_nums_func.pyx 100000000  # (*@$\approx$@*) 14 seconds
		cython_build.sh sum_nums_func.pyx
		time ./sum_nums_func 100000000  # (*@$\approx$@*) 12 seconds
	\end{lstlisting}
\end{frame}

\section{Variable Declaration}
\begin{frame}[fragile]{Static Type Declaration in Cython}
	\begin{lstlisting}
		cdef char c
		cdef unsigned char b
		cdef int i
		cdef long j
		cdef unsigned int k
		cdef unsigned long long l
		cdef float f
		cdef double d
		cdef char* s (*@\pause@*)
		cdef struct  (*@(Maybe talk about this later)@*)
	\end{lstlisting}
\end{frame}

\begin{frame}[fragile]{Declaring the Iterator}
	sum\_nums\_func.pyx
	\begin{lstlisting}
		def sum_nums(n):
		    s = 0
		    (*@\colorbox{green!20}{cdef unsigned long i  \# $\leftarrow$ defines i as a unsigned long}@*)
		    for i in range(n+1):
		    	    s += i
		    	return s
		
		import sys
		n = int(sys.argv[1])
		print(sum_nums(n))
	\end{lstlisting}
	\pause
	\begin{lstlisting}[language=Bash]
		time python sum_nums_func.pyx 100000000  (*@\pause@*)# SyntaxError (*@\pause@*)
		cython_build.sh sum_nums_func.pyx
		time ./sum_nums_func 100000000  # (*@$\approx$@*) 12 seconds (Slightly faster than without the cdef)
	\end{lstlisting}
\end{frame}

\begin{frame}[fragile]{Declaring the Sum}
	sum\_nums\_func.pyx
	\begin{lstlisting}
		def sum_nums(n):
		    (*@\colorbox{green!20}{cdef unsigned long}@*) s = 0
		    cdef unsigned long i  # (*@$\leftarrow$@*) defines i as a unsigned long
		    for i in range(n+1):
		    	    s += i
		    	return s
		
		import sys
		n = int(sys.argv[1])
		print(sum_nums(n))
	\end{lstlisting}
	\pause
	\begin{lstlisting}[language=Bash]
		cython_build.sh sum_nums_func.pyx
		time ./sum_nums_func 100000000  # (*@$\approx$@*) 0.5 seconds (Slightly fasterer than without the cdef)
	\end{lstlisting}
\end{frame}

\begin{frame}[fragile]{Cython Command \& HTML Annotations}
	\begin{lstlisting}[language=Bash]
		cython -a --embed ${cython_file} -o ${c_file}
	\end{lstlisting}
	
	\begin{description}[<+->]
		\item[-o \$\{c\_file\}] Specifies the name of the resulting C file
		\item[--embed] Compiles the C code with a main method 
		\item[-a] Produces a helpful HTML file
	\end{description}
	
	% Pass \thebeamerpauses to pause so that the text will appear one 'slide'
	% after the descriptions. \thebeamerpauses holds the current number of 
	% pauses in the current slide
	\pause[\thebeamerpauses]
	\emptyline
	\emptyline
	sum\_nums\_py.html \& sum\_nums\_cy.html
\end{frame}

\section{Function Declaration}
\begin{frame}[fragile]{Return and Parameter Typing}
	sum\_nums\_func.pyx
	\begin{lstlisting}
		(*@\colorbox{green!20}{cdef}@*) sum_nums((*@\colorbox{green!20}{unsigned long}@*) n):
		    cdef unsigned long s = 0
		    cdef unsigned long i
		    for i in range(n+1):
		    	    s += i
		    	return s
		
		import sys
		n = int(sys.argv[1])
		print(sum_nums(n))
	\end{lstlisting}
\end{frame}

\end{document}